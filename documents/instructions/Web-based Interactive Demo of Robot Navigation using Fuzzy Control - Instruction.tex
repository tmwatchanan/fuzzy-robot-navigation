%%%%%%%%%%%%%%%%%%%%%%%%%%%%%%%%%%%%%%%%%%%%%%%%%%%%%%%%%%%%%%%%%%%%%%
% LaTeX Example: Project Report
%
% Source: http://www.howtotex.com
%
% Feel free to distribute this example, but please keep the referral
% to howtotex.com
% Date: March 2011 
% 
%%%%%%%%%%%%%%%%%%%%%%%%%%%%%%%%%%%%%%%%%%%%%%%%%%%%%%%%%%%%%%%%%%%%%%
% How to use writeLaTeX: 
%
% You edit the source code here on the left, and the preview on the
% right shows you the result within a few seconds.
%
% Bookmark this page and share the URL with your co-authors. They can
% edit at the same time!
%
% You can upload figures, bibliographies, custom classes and
% styles using the files menu.
%
% If you're new to LaTeX, the wikibook is a great place to start:
% http://en.wikibooks.org/wiki/LaTeX
%
%%%%%%%%%%%%%%%%%%%%%%%%%%%%%%%%%%%%%%%%%%%%%%%%%%%%%%%%%%%%%%%%%%%%%%
% Edit the title below to update the display in My Documents
%\title{Project Report}
%
%%% Preamble
\documentclass[paper=a4, fontsize=11pt]{scrartcl}
\usepackage[T1]{fontenc}
\usepackage{times}

\usepackage[english]{babel}															% English language/hyphenation
\usepackage[protrusion=true,expansion=true]{microtype}	
\usepackage{amsmath,amsfonts,amsthm} % Math packages
\usepackage[pdftex]{graphicx}	
\usepackage{url}
\usepackage{hyperref}


%%% Custom sectioning
\usepackage{sectsty}
\allsectionsfont{\normalfont\scshape}


%%% Custom headers/footers (fancyhdr package)
\usepackage{fancyhdr}
\pagestyle{fancyplain}
\fancyhead{}											% No page header
\fancyfoot[L]{}											% Empty 
\fancyfoot[C]{}											% Empty
\fancyfoot[R]{\thepage}									% Pagenumbering
\renewcommand{\headrulewidth}{0pt}			% Remove header underlines
\renewcommand{\footrulewidth}{0pt}				% Remove footer underlines
\setlength{\headheight}{13.6pt}


%%% Equation and float numbering
\numberwithin{equation}{section}		% Equationnumbering: section.eq#
\numberwithin{figure}{section}			% Figurenumbering: section.fig#
\numberwithin{table}{section}				% Tablenumbering: section.tab#


%%% Maketitle metadata
\newcommand{\horrule}[1]{\rule{\linewidth}{#1}} 	% Horizontal rule

\title{
	%\vspace{-1in} 	
	\usefont{OT1}{bch}{b}{n}
	\normalfont \normalsize \textsc{IEEE CIS Student Competition 2017-Edition: \\ "Telling a Story: How your Computational Intelligence Research benefits Society and Humanity"} \\ [25pt]
	\horrule{0.5pt} \\[0.4cm]
	\huge Web-based Interactive Demo of Robot Navigation using Fuzzy Control 
	\\\textbf{Instruction}
	\\
	\horrule{2pt} \\[0cm]
}
%\author{
%	\normalfont 								\normalsize
%	Watchanan Chantapakul\\[-3pt]		\normalsize
%	\today
%}
%\date{November 11, 2017}
\date{}

%%% Begin document
\begin{document}
	\maketitle
	
	\section{Details}
	
%	\subsection{Student Information}
	\begin{itemize}
		\item Online demo: \href{https://tmwatchanan.github.io/fuzzy-robot-navigation/index.html}{Web Demo}
		\item Link of this document: \href{https://github.com/tmwatchanan/fuzzy-robot-navigation/raw/master/documents/Web-based%20Interactive%20Demo%20of%20Robot%20Navigation%20using%20Fuzzy%20Control.pdf}{Demo Instruction}
		\item Name of Student: \textbf{Watchanan Chantapakul}
		\item Education Institution:\textbf{Chiang Mai University}
		\item Supervisor: \textbf{Sansanee Auephanwiriyakul}
		\item Type of Artefact: \textbf{Interactive Tutorial / Demo}
	\end{itemize}
	
%	\subsection{Artefact Information}	
	
	
	\section{Introduction}
	This work aims to develop a web-based application which show the way robot navigation system works interactively. Thus, the demo was designed to be built with web technology. It enables us to be more accessible from everywhere via internet. HTML and CSS are the main languages used to structuring web pages and defining styles, respectively. Moreover, JavaScript --- a programming language which adds interactivity to web pages, is inserted in each part of an HTML document.
	
	\section{Instruction}
	Firstly, you needs to open \href{http://tmwatchanan.github.io/fuzzy-robot-navigation}{the url of online demo} mentioned above with your browser. You will see the home page, which shows summarized details about this project and content. The navigation bar placed on the top of every pages makes the user navigate through each different page. It includes links to \textit{(1) Home} \textit{(2) Demo} \textit{(3) Membership Functions}, and \textit{(4) About}. For the last one, after clicking the button, a dialog will be popped up to shows author's information.
	\linebreak \linebreak
	Robot navigation simulation is included on the left side of \textit{Demo page}. On the right side on this page, fuzzy rules are also displayed in form of tables. They are grouped into four tables in according to its output. You can play with the robot inside the square canvas provided in size $600 \times 600$ pixels. 
	
	\begin{thebibliography}{9}
		\bibitem{knuthwebsite} 
		Knuth: Computers and Typesetting,
		\\\texttt{http://www-cs-faculty.stanford.edu/\~{}uno/abcde.html}
	\end{thebibliography}
	
	
	%%% End document
\end{document}